A coaxial HPGe detector was used to collect spectroscopic data from calibration sources of various energies (${}^{241}$Am, ${}^{137}$Cs, ${}^{60}$Co). In addition to a gamma source, a pulser input was also injected into the preamplifier in each measurement to provide information about electronic noise. 

Signal digitization was done using a Struck Integrated Systems 3302 DAQ module. This module has an internal 100 MHz clock which was used for all of the measurements. For this work, preamplifier output pulses were fed directly to the DAQ module without additional processing. The preamplifier output pulses were digitized and used to perform this analysis.The trigger threshold value was set so that gamma-ray pulses of all of the energies of interest could be collected without excessive noise and without exceeding the voltage range of the DAQ.

The SIS3302 module has signal shaping capabilities. There is an on-board 'slow' trapezoidal filter that can be used to collect spectroscopic information and a 'fast' trapezoidal filter used for event triggering. There is also an on-board 'fast' trapezoidal filter used for event triggering. The 'fast' filter was used to reject events with more than one trigger per event. Here that corresponds to a very small ($>1\%$) fraction of events.

Using the digital trapezoidal filter discussed above, spectroscopic information was extracted using variable gap and peaking times. At first, a peaking time of 500 ns was set and the gap time was varied from 1 to 1000 ns. Then, using the optimal gap time, peaking time was similarly varied from 100 ns to 20 $\mu$s. The amplitudes of the trapezoids were used to generate spectra. A region of interest was selected around each peak by visual inspection. Then the region was fit with a Gaussian function and a linear function using the built in models in LmFit \cite{LMFIT}. The FWHM of the Gaussian was extracted and used. Gamma energy values were taken from Table 12.1 in \cite{Knoll} .