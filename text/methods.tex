A coaxial HPGe detector was used to collect spectroscopic data from calibration sources of various energies (${}^{241}$Am, ${}^{137}$Cs, ${}^{60}$Co). In addition to a gamma source, a pulser input was simultaneously injected into the preamplifier in each measurement to provide a measure of electronic noise. 

A Struck Integrated Systems (SIS) 3302 DAQ module. This module has 16-bit ADCs and an internal 100 MHz clock which was used for all of the measurements. Preamplifier output signals were fed directly to the DAQ module without additional processing. These output signals were digitized and used to perform this analysis.

The trigger threshold value was set so that gamma-ray pulses of all of the energies of interest could be collected without excessive noise and without exceeding the voltage range of the DAQ. The SIS 3302 module has signal shaping capabilities. There is an on-board 'slow' trapezoidal filter that can be used to collect spectroscopic information and a 'fast' trapezoidal filter used for event triggering. 

Data was collected in a way to ensure that pile-up effects would be negligible. The event rate was chosen to be approximately 200 Hz for a system with a 100 MHz clock (10 ns sampling time) and a preamplifier pulse decay time of roughly 500 ns,. Still, the 'fast' filter was used to reject events with more than one trigger per event. Here that corresponds to a very small fraction of events ($<1\%$). 

Before further processing each pulse was baseline corrected. This was done by subtracting the average value of the first 800 samples (for each event, the rise of the preamplifier signal begins at approximately the 1000 th sample) of a pulse from all of the samples in that pulse.

Next, the raw signals were fit with exponential functions to extract the decay time constant (M). The average decay time constant for all of the ${}^{137}$Cs signals (without multiple triggers) was used as M. To first order M is the RC time constant of the preamplifier circuit. This does not depend on energy and should be the same for all measurements taken with the detector.

A digital filter was implemented following the description in Figure 5 of \cite{Jordanov}. Using the filter, spectroscopic information was extracted using variable gap and peaking times. First, a peaking time of 500 ns was set and the gap time was varied from 0 to 1000 ns. Then, using the optimal gap time, peaking time was similarly varied from 100 ns to 20 $\mu$s. Once the filter was optimized, the amplitudes of the trapezoids were used to generate spectra. A region of interest was selected around each peak by visual inspection. Then the region was fit with a Gaussian function and a linear function using the built in models in LmFit \cite{LMFIT}. The FWHM of the Gaussian was extracted and used. 

To optimize parameters, the 662 keV peak from $^{137}$Cs was used, since it is roughly in the center of the energy range studied. However, data from three other peaks was also examined (${}^{241}$Am 59.536 keV, ${}^{60}$Co 1173.231 keV , ${}^{60}$Co 1332.508 keV) and gave similar results for parameter optimization. Gamma energy values were taken from Table 12.1 in \cite{Knoll} .

Using the optimal parameters for $^{137}$Cs (a gap time of 550 ns and a peaking time of 11.64 $\mu$s) the energy resolution of the system as a function of gamma energy was studied.

oisson's equation in cylindrical coordinates can be written in the following way:

\begin{align}\label{eq:maxwell}
\nabla ^2\phi = \left( {\underbrace {\frac{\partial ^2\phi
}{\partial r^2} + \frac{1}{r}\frac{\partial \phi }{\partial
r}}_{\frac{1}{r}\frac{\partial \phi }{\partial r}\left(
{r\frac{\partial \phi }{\partial r}} \right)} +
\frac{1}{r^2}\frac{\partial ^2\phi }{\partial \theta ^2} +
\frac{\partial ^2\phi }{\partial z^2}} \right) = -\frac{\rho}{\varepsilon_\mathrm{o}}
\end{align}

\noindent where $\phi$ is the potential and $\rho$ is the total space charge density, which will be taken as a constant. For a cylindrically symmetric configuration, the azimuthal partial derivative term can be taken as zero. For a tractable closed form solution of the electric field inside the coaxial HPGe detector, it is necessary to ignore the end cap region. The electric field flux near the corners in particular will lead to irregular pulse signals and so in this analysis will be ignored. In the regions away from the end, this allows the z term to be dropped as well. Equation \ref{eq:maxwell} now becomes:
\begin{align}\label{eq:maxsimplified}
\frac{\partial ^2\phi
}{\partial r^2} + \frac{1}{r}\frac{\partial \phi }{\partial
r} = -\frac{\rho}{\varepsilon}
\end{align}
\indent Since there is only an r dependence, the electric field is also only in the radial direction or E(r) = $-\partial\phi/\partial r$. This turns equation \ref{eq:maxsimplified} into the following form:
\begin{align}
\nabla_{r} \cdot \vec{\mathrm{E}}\mathrm{(r)} = \frac{\partial \mathrm{E(r)}
}{\partial r} + \frac{\mathrm{E(r)}}{r} = \frac{\rho}{\varepsilon}
\end{align}

For a fully depleted HPGe coaxial detector with the added boundary condition that the the voltage between the inner and outer radius is the applied voltage, it can be shown that:

\begin{align}
|E(r)| &= -\frac{\rho}{2\varepsilon}r + \frac{V - V_{D}}{r \mathrm{ln}\left(\frac{b}{a}\right)}\\
V_{D} &= \frac{\rho (b^{2} - a^{2})}{4 \varepsilon} \nonumber
\end{align}

Where b is the outer radius, a is the inner radius, and $V_{D}$ is the depletion voltage.
\subsection*{Shockley-Ramo Approximation: Ignore All Space Charge}
Applying the Shockley-Ramo theorem \cite{ramo,Shockley}, the space charge $\rho$ is set equal to zero and the applied voltage V is set to unity. This gives for the weighting field:

\begin{align}
    |E_{w}(r)| = \frac{1}{r\mathrm{ln}\left(\frac{b}{a}\right)}
\end{align}

The induced current on the electrodes become:

\begin{align}
    i = q E \cdot \vec{v} = q E_{v} v
\end{align}

Where q is the charge and $E_{v}$ is the component of the electric field in the direction of the instantaneous velocity v of an electron. It is assumed that the applied potential is strong enough to make diffusion effects negligible and constrain the paths of the formed electrons and holes to radial lines. Since there are only two conducting surfaces this means that the weighting field and the actual electric field are proportional to one another (as seen above) so that:

\begin{align}
    i = \frac{qv}{r\mathrm{ln}\left(\frac{b}{a}\right)}
\end{align}

The drift velocity of an electron or hole in a semiconductor is found in solid state physics to be proportional to the electric field provided that the field is not too large i.e.:

\begin{align}\label{velocity}
    v_{e/h} = \mu_{e/h} E
\end{align}

Where $\mu$ is the electron/hole mobility: a constant of the material for our operation space. For single interaction events at a fixed radius, $r_{o}$, a total number of charges n may be generated locally near the interaction site as a result of indirect ionizations. The electrons in an n-type germanium detector will move radially inward, the holes radially outward. For the electrons (similar holds for the holes with opposite sign):
\begin{align}
    v_{e} &= \mu_{e} \frac{V}{r_{e}\mathrm{ln}\left(\frac{b}{a}\right)}= \frac{dr_{e}}{dt} \nonumber \\
    \rightarrow  r^{2}_{e}(t) &= r_{o}^{2} + \frac{2\mu_{e}V}{\mathrm{ln}\left(\frac{b}{a}\right)} t \\
    r_{e}^{2}(t) &= r_{o}^{2} \left(1 + \frac{t}{t_{o}} \right) \label{radius} \\
    \frac{r_{e}(t)}{r_{o}} &=\left(1 + \frac{t}{t_{o}} \right)^{\frac{1}{2}} 
\end{align}

Where:
\begin{align}
    t_{o} = \frac{r_{o}^{2} \mathrm{ln}\left(\frac{b}{a}\right)}{2 \mu_{e} V}
\end{align}

The induced charge can be found by integrating the induced current. For electrons:

\begin{align}
    Q_{e}(t) &= \int_{0}^{t}i_{e}(t')dt' = \frac{q_{e}}{\mathrm{ln}\left(\frac{b}{a}\right)}\int_{r_{o}}^{r_{e}(t)}\frac{v_{e}}{r}\frac{dr}{v_{e}}\nonumber \\
   &= \frac{q_{e}}{\mathrm{ln}\left(\frac{b}{a}\right)}\mathrm{ln}\left(\frac{r_{e}(t)}{r_{o}}\right) = \frac{q_{e}}{2\ \mathrm{ln}\left(\frac{b}{a}\right)}\mathrm{ln}\left(1 + \frac{t}{t_{o}}\right)
\end{align}
The total induced charge is just the sum of the holes and the electrons:

\begin{align} \label{induced}
     Q_{\mathrm{tot}}(t) =  \frac{ne}{2\ \mathrm{ln}\left(\frac{b}{a}\right)} \left[\mathrm{ln}\left(1 - \frac{t}{t_{o,h}}\right) + \mathrm{ln}\left(1 + \frac{t}{t_{o,e}}\right)\right]
\end{align}

Where $t_{o,e}$ is for electrons and $t_{o,h}$ is for holes. This shape assumes that there will none of either charge species is lost or trapped. There will also, in general, be a discontinuous slope change when the first of the electrons or holes arrives at their respective collecting electrodes. These collection times $T_{e,h}$ depend on $r_{o}$ from Equation \ref{collect}:

\begin{equation}
\begin{aligned} \label{times}
  T_{h} &= \frac{(r_{o}^{2} - a^{2})}{r_{o}^2}t_{o,h} = \left(1 - \frac{a^2}{r_{o}^2} \right)t_{o,h}\\
  T_{e} &= \frac{(b^{2} - r_{o}^{2})}{r_{o}^2}t_{e,h} = \left(\frac{b^2}{r_{o}^2}-1 \right)t_{e,h}
\end{aligned}
\end{equation}

\subsection*{More General Case}

Starting from equation \ref{eq:maxsimplified} there will be a space charge $\rho$ associated with residual donor impurities that extends all the way from radius a out to the outer radius b in a fully depleted detector. A thin layer extends there that has the same but opposite magnitude of charge and contributes negligibly to the potential from the inner to outer radius. An applied voltage V will in turn induce a surface charge $\lambda$ on the inner conductor (and $-\lambda$ on the outer conductor). Thus using Gauss's Law with a cylinder of radius r and length l:

\begin{align}
    \int \vec{E}(r) \cdot d\hat{A} &= E_{r}2\pi r l = \frac{\lambda l + \pi \rho l (r^2 - a^2)}{\varepsilon} \label{electric_field}\\
    \rightarrow \varepsilon E_{r} &= \frac{\lambda}{2\pi r} + \frac{\rho}{2r}(r^2-a^2) \label{ewithlambda}
\end{align}

When integrating both sides from r = a to r = b, the left hand side becomes -V (reverse biased). This boundary condition fixes $\lambda$:

\begin{align}
    \int_{a}^{b} E_{r}dr &= -V = \frac{\lambda \mathrm{ln}(b/a)}{2 \pi\varepsilon} + \frac{\rho(b^2-a^2)}{4\varepsilon} - \frac{a^2}{\varepsilon}\mathrm{ln}(b/a)\\
    \rightarrow \frac{\lambda}{2\pi\epsilon} &= -\frac{V + \frac{\rho}{4\varepsilon}(b^2-a^2)}{\mathrm{ln}(b/a)} + \frac{\rho a^2}{2\varepsilon} \label{lambdasolved}
\end{align}

Plugging equation \ref{lambdasolved} into equation \ref{ewithlambda}, noting that $\rho = -e N_{d}$ where $N_{d}$ is the donor impurity in the n-type HPGe, and taking the absolute value:

\begin{align}
    |E(r)| = \frac{e N_{d}}{2 \varepsilon} r + \frac{V-\frac{e N_{d}}{4\varepsilon}(b^2-a^2)}{r \mathrm{ln}(b/a)}
\end{align}

The induced charge can be found by following the logic of Glenn Knoll.\cite{knoll} The electric field can be recast as: 

\begin{align}
    |E(r)| &= 2\alpha r + \frac{\beta}{r} \label{e_simplified}\\
    \alpha &= \frac{e N_{d}}{4\varepsilon} \nonumber\\
    \beta &= \frac{V-\frac{e N_{d}}{4\varepsilon}(b^2-a^2)}{\mathrm{ln}(b/a)} \nonumber
\end{align}

Integration of the energy lost by the electrons and holes as they travel to the electrodes from an initial interaction point $r_{o}$ in turn gives a relationship for the induced charge on the collecting electrodes since the applied voltage is fixed:

\begin{align}
    Q(t) &= Q_{-}(t) + Q_{+}(t)\\
    Q_{-}(t) &= \frac{q_o}{V} \left(\alpha (r_{o}^2 - r_e^2(t)) + \beta \left(\mathrm{ln}\frac{r_{o}}{r_{e}(t)}\right)\right)\\
     Q_{+}(t) &= \frac{q_o}{V} \left(\alpha (r_{h}^2(t) - r_{o}^2) + \beta \left(\mathrm{ln}\frac{r_{h}(t)}{r_{o}}\right)\right)
\end{align}

Using equation \ref{velocity}, it is possible to find the collection times similar to equation \ref{times}:

\begin{align}
    v_{d} = \frac{dr_{h/e}}{dt} &= \pm \mu_{h/e} E = \pm \mu_{h/e} (2\alpha r + \frac{\beta}{r})\\
    \rightarrow \int_{r_{o}}^{r_{h/e}(t)} \frac{dr}{2\alpha r + \frac{\beta}{r}}&= \pm \int_{0}^{t} \mu_{h/e} dt 
\end{align}

Where plus is taken for holes, minus for electrons. The radial position of holes and electrons can be solved with algebra:

\begin{align}
    r_{h}^2(t) &= \left(r_{o}^2 + \frac{\beta}{2\alpha}\right)e^{4\alpha \mu_{h} t} - \frac{\beta}{2 \alpha}\\
    r_{e}^2(t) &= \left(r_{o}^2 + \frac{\beta}{2\alpha}\right)e^{-4\alpha \mu_{e} t} - \frac{\beta}{2 \alpha}
\end{align}

The collection times are when $r_{e}(t_c) = a$ and $r_{h}(t_c) = b$. This gives with some algebra:

\begin{equation}
\begin{aligned}\label{times_general}
  T_{h} &= \mathrm{ln}\left(\frac{b^2 + \omega}{r_{o}^2 + \omega} \right)/(4 \alpha \mu_{h})\\
  T_{e} &= -\mathrm{ln}\left(\frac{a^2 + \omega}{r_{o}^2 + \omega} \right)/(4 \alpha \mu_{e})\\
  \omega &= \frac{\beta}{2\alpha} \nonumber
\end{aligned}
\end{equation}

These equations were used to model signal shapes within the detector given the following specs of the HPGe used here as well as values in Knoll:\cite{knoll} 

\begin{align}
    a &= \frac{0.91\ \mathrm{cm}}{2}\nonumber \\
    b &= \frac{5.23\ \mathrm{cm}}{2} \nonumber \\
    \varepsilon &= 16\varepsilon_{o} \nonumber \\ \mu_{h}(70K) &=  3.6 \times 10^{4} \frac{\mathrm{cm^2}}{Vs} \nonumber \\
    \mu_{e}(70K) &=  4.2 \times 10^{4} \frac{\mathrm{cm^2}}{Vs}  \nonumber \\
   \rho &= 10^{10} \frac{\mathrm{impurities}}{\mathrm{cm}^3}
\end{align}

The value for $\rho$ is taken as a best guess.

