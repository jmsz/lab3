Data was provided by Dr. Ross Barnowski and consisted of counts as a function of channel number for 5 different isotopes (${}^{137}$Cs, ${}^{241}$Am, ${}^{133}$Ba, ${}^{60}$Co, and ${}^{152}$Eu). Gamma-rays were measured with a coaxial HPGe detector. Values were digitized with a 13-bit resolution MCA, giving 8192 channels.

Data was loaded and verified by comparing the md5 checksum provided for the data to that calculated locally.
Two peaks were used to perform the energy calibration: 59.541 keV from ${}^{241}$Am, and 661.657 keV from ${}^{137}$Cs. A narrow region of interest containing each peak was selected through visual examination of the spectra. In this region the scipy.optimize.curve\_fit function was used to fit a gaussian function to the data. Initial fit parameters used were rough estimates determined through visual examination of the data. Figure \ref{AmCsFit} shows the data, initial guess, final fit, and centroid for the ${}^{241}$Am and ${}^{137}$Cs peaks used for the calibration.

\begin{figure}
\centering
\begin{minipage}{.5\textwidth}
  \centering
  \includegraphics[width=1\linewidth]{AmFit.pdf}
\end{minipage}%
\begin{minipage}{.5\textwidth}
  \centering
  \includegraphics[width=1\linewidth]{CsFit.pdf}
\end{minipage}
\caption{This illustrates the gaussian peak fitting which was done for the two peaks of interest. On the left is that for ${}^{241}$Am, on the right that for ${}^{137}$Cs. In each plot, the data is shown with blue circles, the initial guess with green dashed lines, the final fit with a solid black line, and the extracted centroid with a vertical red line.}
\label{AmCsFit}
\end{figure}

The centroids of the fit functions were extracted and used to find coefficients for a linear expression correlating channel number and energy.

The energy calibration was then applied to the ${}^{133}$Ba, ${}^{60}$Co, and ${}^{152}$Eu data to calculate the energies of several peaks. A comparison of true and calculated energies is shown in the results section.
