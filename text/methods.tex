A coaxial HPGe detector was used to collect spectroscopic data from calibration sources of various energies (${}^{241}$Am, ${}^{137}$Cs, ${}^{60}$Co). In addition to a gamma source, a pulser input was also injected into the preamplifier in each measurement to provide information about electronic noise. 

Signal digitization was done using a Struck Integrated Systems 3302 DAQ module. This module has 16-bit ADCs and an internal 100 MHz clock which was used for all of the measurements. For this work, preamplifier output pulses were fed directly to the DAQ module without additional processing. The preamplifier output pulses were digitized and used to perform this analysis.The trigger threshold value was set so that gamma-ray pulses of all of the energies of interest could be collected without excessive noise and without exceeding the voltage range of the DAQ.

Data was collected in ensuring that pile-up effects would be negligible. The event rate was made to be approximately 200 Hz (calibration source and pulser events) for a system with a 100 MHz clock (10 ns sampling time), a preamplifier pulse decay time of roughly 500 ns, and a total sampling time length of roughly 40 $\mu$s per event.

The SIS3302 module has signal shaping capabilities. There is an on-board 'slow' trapezoidal filter that can be used to collect spectroscopic information and a 'fast' trapezoidal filter used for event triggering. There is also an on-board 'fast' trapezoidal filter used for event triggering. The 'fast' filter was used to reject events with more than one trigger per event. Here that corresponds to a very small ($>1\%$) fraction of events. 

Before further processing each pulse was baseline corrected by subtracting the average value of the first 800 samples from the rest. For each event the rise of the preamplifier signal begins at spproximately the 1000th sample. This ensures that the baseline correction does not interfere with signal shapes. 

Next, the raw signals were fit with exponential functions to extract the decay time constant (M). The average decay time constant for all of the ${}^{137}Cs$ signals (without 'retriggering') was used as M. To first order M is the RC time constant of the preamplifier circuit. This does not depend on energy and should be the same for all measurements taken with the detector.

A digital filter was implemented following the description in Figure 5 of \cite{Jordanov}. Using the digital trapezoidal filter discussed above, spectroscopic information was extracted using variable gap and peaking times. At first, a peaking time of 500 ns was set and the gap time was varied from 1 to 1000 ns. Then, using the optimal gap time, peaking time was similarly varied from 100 ns to 20 $\mu$s. The amplitudes of the trapezoids were used to generate spectra. A region of interest was selected around each peak by visual inspection. Then the region was fit with a Gaussian function and a linear function using the built in models in LmFit \cite{LMFIT}. The FWHM of the Gaussian was extracted and used. Gamma energy values were taken from Table 12.1 in \cite{Knoll} .


To optimize parameters, the 662 keV peak from $^{137}$Cs was used, since it is roughly in the center of the energy range studied. However, data from three other peaks was also examined (${}^{241}Am$ 59.536 keV, ${}^{60}Co$ 1173.231keV , ${}^{60}Co$ 1332.508 keV) and gave very similar results.

