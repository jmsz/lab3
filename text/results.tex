The energy calibration was found to be described by:
\begin{equation}
Energy = 0.28054 * Channel + 1.26366
\end{equation}

The true and calculated energies of ${}^{133}$Ba are compared in Table 1. The last column of the table shows the percent error in our calculated energies. A plot showing the calibrated spectrum with the calculated centroids is shown in Figure \ref{Ba133spectrum}. The centroids match well with the peaks.

\input{./images/Ba133_table.txt}

\begin{figure}[h]
\begin{centering}
\includegraphics[width=0.8\textwidth]{CalibratedBa133.pdf}
\caption{A calibrated spectrum for ${}^{133}$Ba}
\label{Ba133spectrum}
\end{centering}
\end{figure}
\vspace{5mm}

The same analysis was done for ${}^{60}$Co and ${}^{152}$Eu. The relevant values and in Table 2 and Table 3. The percent error is small and does not appear to be energy-dependent. Thus, the energy calibration works well for the peaks studied here. The calibrated spectra for these two isotopes are shown below in Figure \ref{Co60spectrum} and Figure \ref{Eu152spectrum}. The centroids of the studied peaks are indicated with red lines.

\begin{figure}[h]
\centering
\includegraphics[width=0.8\textwidth]{CalibratedCo60.pdf}
\caption{A calibrated spectrum for ${}^{60}$Co}
\label{Co60spectrum}
\vspace{-4.5mm}
\end{figure}

\begin{figure}
\centering
\includegraphics[width=0.8\textwidth]{CalibratedEu152.pdf}
\caption{A calibrated spectrum for ${}^{152}$Eu}
\label{Eu152spectrum}
\vspace{-5.0mm}
\end{figure}