In this work, a digital trapezoidal filter was implemented for gamma-ray spectroscopy with a coaxial HPGe detector. A digital rather than analog implementation allows for greater ease and flexibility in tuning parameters. The two parameters of the filter that can be optimized are the peaking time and the gap time. Often in applications a trade-off between rate performance and energy resolution must be made.  Here the filter is being tuned to achieve the best possible energy resolution, disregarding rate performance. Then, using the optimized filter on spectral data from several calibration sources, the Fano factor was calculated.

In common HPGe detector configurations, a charge-sensitive preamplifier integrates the charge induced on an electrode and outputs a voltage proportional to the collected charge. A feedback resistor and capacitor provide an RC circuit which causes an exponential decay of the preamplifier's output signal. These pulses are shaped before further processing to eliminate their long decay tail while preserving information about signal amplitude, timing and signal shape.

Although theoretically cusp or triangular filters give better noise performance, trapezoidal filters are more commonly used. This is due partially to the impracticality of digitizing a signal with a small flat top. Additionally, the trapezoidal filter is advantageous for detectors which see variable charge collection time as the flat top can be tuned to avoid ballistic deficit \cite{Knoll}.

Due to the random nature of radioactive decay, even at low count rates pile-up must be considered. To prevent pile-up, shaped pulses should be kept short, minimizing peaking and gap times. However, making these parameters too short can lead to incomplete charge collection or poor noise performance. Making these parameters too long can lead to pile-up or poor noise performance.

This is because of the dependance on peaking time of the three main contributions to the electronic noise- series noise, parallel noise and 1/$f$ noise. The series noise $\propto \frac{1}{\tau}$. The parallel noise is $\propto\tau$. The 1/$f$ noise is independent of $\tau$. Thus a balance between the noise contributions must be found for some optimal $\tau$. 

