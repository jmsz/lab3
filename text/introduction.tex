
Energy calibration of detectors is a crucial part of gamma-ray spectroscopy. How well we can determine energies depends partially on the quality of energy calibration. Accurate determinations of the energies lessen errors in measurements. This can be crucial for determining the origin or path of a gamma ray. This information is used to determine from which physical places and processes gamma rays originated in, the nature of these processes, the properties of other participants or surrounding structures.

The purpose of this experiment is to demonstrate a linear energy calibration using provided spectral data. Peaks are fit with a gaussian function to determine the peak centroid, amplitude, and width. These parameters give valuable information. The width of the peak is correlated with the energy resolution of the measurement. The amplitude can be useful for determining efficiency of the measurement, branching ratio of a certain transitions, various noise contributions of the system, etc.\cite{Knoll:1300754}

Here, the centroid is the parameter of interest. This value gives us the most likely channel number corresponding to the energy of the gamma ray. The centroid channel values and the true gamma ray energies are used to calculate a linear fit function that relates channel number and energy. This function comprises our energy calibration.
