\subsection*{Position Sensing in HPGe Detectors}

In many applications of HPGe detectors, interaction position information is necessary. One example of such an application is Compton imaging. The precision to which the interaction locations are known will effect the precision and accuracy of the reconstructed image.
Double-sided strip detectors (DSSDs) have 3D position sensitivity. The collecting electrodes on one face are orthogonal to the collecting electrodes on the other face. Thus, by determining which strips on each side have collected charge from an interaction, one can extract 2D position information. Most DSSDs have 2D position resolution finer than their strip pitch. This is can be achieved through various methods utilizing pulse shape analysis. Looking at the shape of the rising edge of preamplifier signals can provide information about an event - how many interactions took place, where in the detector, and with what energy deposition. 
The third dimension, the distance of the interaction from either face of the detector, or the depth of the interaction, can determined by pulse-shape analysis techniques. The difference in arrival times between the signals on the two faces of the detector gives information about the depth. If the interaction took place near the cathode, the weighting potential of the cathode will be stronger in this region and the signal from charge carrier motion in the crystal will be seen sooner here than at the anode. Here, simple pulse shape analysis techniques are used to determine the depths of interactions in a DSSD.

\subsection*{CCI2}

The CCI2 ("Compact Compton Imager II") was used for this work. This system consists of two planar HPGe detectors stacked in a cryostat with a $\approx$ 10 mm distance between them. Each detector is roughly square in shape with a width of 74 mm, and a height of 15.1 mm. The collecting electrodes have a 2 mm pitch, the strips being separated by a 0.25 mm gap. Further details of the system can be found in \ref{cci21} and \ref{cci22}.