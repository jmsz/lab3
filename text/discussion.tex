The fitting could be improved to deal with more complex shapes. The simple linear background assumption does not work well for all of the peaks of interest (particularly in the case of  ${}^{241}Am$). Additionally, at very short rise and gap times, peaks do not always resemble Gaussians and could be better represented with skewed Gaussians or other shapes. 


The calculated Fano factor is rather low and has significant error from fitting. An overestimate of the statistical noise would lead to a low Fano factor value. Indeed, since we only substract a measured value of electronic noise, there could be other noise in our system (potentially from charge loss in the detector) which is not accounted for, leading to this discrepancy.

We could measure charge loss in our system by inputing a pulse of known charge and voltage through the detector we could determine detector capacitance. The discrepancy in measured charge from a pulse input to the detector and a pulse input to the preamplifier could possibly provide a measure for charge loss.

Many improvements could be made to the current implementation of the trapezoidal filter and to the system. For each signal the amplitude was extracted by taking the maximum value of the trapezoidal signal. This is not the best approach and may introduce some error. Due to the presence of noise in the flat-top, the extracted amplitudes will likely be biased towards higher values. Additionally, using this method, it is very important to have a good pole-zero correction. If the flat-top of the signal is not flat but sloped, as happens with an incorrect time constant, the correct amplitudes will not be extracted. Depending on the amount of noise in the system, sampling rate, gap time, and pole-zero correction quality it may be desirable to implement a better strategy. Other strategies that could be implemented include: sampling at the midpoint of the filter or averaging a few samples in the flat-top. Additionally, a measure of the flatness of the flat-top could be used as a measure of event quality.

Looking at FWHM/ FWTM ratio could give additional info on ballistic deficit
