More complex fitting can be done to calibrate spectra (high oder polynomial fits, various fits over the energy ranges of interest, etc.) \cite{Knoll:1300754}. Here a simple linear fit using two measurements was done. The quality of this simple calibration is adequate. The error given by the energy calibration is much less than 0.2$\%$ over the energy range examined, and so is less than the expected contributions from other sources of error in a typical HPGe detector system.  Additionally, the quality of the calibration does not seem to depend on gamma ray energy, and so t should not bias our measurements.

Although these spectra have neat, thin, well-separated gaussian peaks, it may be of interest to try using more complex fitting functions. For example, some data sets studied here (particularly ${}^{241}$Am), may have been better fit with a gaussian and a step function to account for lower-energy background near the peak. Including a step function or linear function to describe the background on spectra may help us get more precise fits. For convoluted spectra, more complex fitting and background subtraction would likely be needed to achieve similar accuracy.
